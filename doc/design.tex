
\chapter {System Design}


\section{Overall Design}

HAPI has been designed to be a fully modular haptic rendering engine,
allowing users to easily substitute or modify any component of the
haptics rendering process. HAPI is primarily targetted at 3DOF and
6DOF point based haptics rendering systems.



\subsection{Device Abstraction Layer}
The Device Abstraction Layer (DAL) provides a device independent
interface for various haptics devices. The DAL covers the following
areas:

\begin{itemize}
\item Device Initialisation
\item Get Position / Orientation 
\item Set Force / Torque
\item Error management
\item Device safety (max force, temperature, etc)
\end{itemize}

A DAL node is required for each haptics device to be supported by HAPI.


\subsection{Thread Layer}
The Thread Layer (TL) will provide simple real-time scheduling
mechanisms for devices in the DAL. TL should allow for 1000Hz
scheduling as a minimum, with options for higher or lower real-time
rates.

TL will also provide for thread synchronisation mechanisms between the
real-time thread(s) and other threads.  Asyncrhonous callbacks will be
necessary to prevent extended locking of the real-time threads.


\subsection{Proxy Layer}
The Proxy Layer (PL) is a flexible abstraction of proxy concepts. The
base proxy class contains only the following members:

\begin{itemize}
\item {\it position}
\item {\it orientation}
\end{itemize}

The design of HAPI is such that the other higher layers should be able
to handle a point proxy, a spherical proxy and a proxy of arbitrary
geometry or design that inherit from this base class.

The default proxy, SphericalProxy, is a finite sized sphere that also
contains a {\it radius} member.

The {\it orientation} member can be used to affect surface friction
and is useful with 6DOF devices.

The PL also defines how a proxy should be connected to a haptics
device. The proxy/device model of PL supports all combinations of
connections; one to one, one to many, many to one and many to
many. Just as one device can be connected to several proxies, one
proxy can be connected to several devices. The default connection is a
simple spring model, but this can be replaced by a custom model. The
only requirement is that the connection must convert a proxy position
and device position into a force to be rendered on that device.


\subsection{Collision Detection Layer}
The Collision Detection Layer (CDL) is called in the scene-graph loop
to determine possible collision with a particular geometry.

Will contain default classes for basic geometry primitives and
triangle meshes.

The CDL class is given a proxy node and a movement vector. If the CDL
class detects a potential collision then it returns a simplified
version of the geometry in the region of the proxy.


\subsection{Realtime Collision Layer}

The Relatime Collision Layer (RCL) runs, as its name indicates, in the
real-time loop. The RCL performs collision detection between proxy and
the simplified geometry. Collisions generate surfaces to be rendered
by the Surface Rendering Layer.

The RCL works with simplified geometries, or implicit surfaces, and
will require one node for each geometry type and proxy
type. E.g. RCL for triangle mesh with spherical proxy.


\subsection{Surface Rendering Layer}
(TODO check this subsection)

The Surface Rendering Layer (SRL) will update the position of the
proxy based on previous proxy position and current finger
position. SRL works in surface texture coordinate space.

\begin{itemize}
\item proxy layering
\item multiple proxies?
\item magnetic/sticky surfaces
\item force shading?
\item moving geometries / dynamic o transform 
\item forcemodels adding constraint planes? VHTK proxy
\item OpenGL feedbackbuffer triangle meshes
\item probably need double precision, but should be able to choose
non-uniform scaling? problem?
\end{itemize}

Surface has a local coordinate system.
Surface - input: position, orientation, velocity, angular\_velocity,
texture coordinate, tc derivatives with respect to local coordinates
         output: force, torque?, jacobian matrices for force dependent
         on pos/vel, proxy movement (for HapticMaster support)


HapticAlgorithm bindable node( or per geometry? should you be able to
have different algorithms for different geometry? )
- settings for different haptic algorithms.
- determines what algorithm to use.

Contact abstraction:
- point, normal, triangle? , 
- each RCO(=??) has a list of current contacts
- contacts are updated each haptics loop with new values.

Haptic thread:
Collide proxy with RCO and generate forces, torques, etc. Different
rendering algorithms possible. Implemented from the beginning:


Modified Ruspini - 
\begin{itemize}
\item 1. Find constraint planes for all geometries.
\item 2. Set the goal of the proxy to be the finger and force to be 0
\item 3. Move proxy towards goal - 
  - if no new collision proxy moves to goal and force is returned
  - closest collision - call surface/s to get force and proxy movement.
  - set goal to be proxy + proxy\_movement 
  - repeat 3
\end{itemize}


 


