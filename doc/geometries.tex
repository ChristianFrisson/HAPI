\chapter{Collision geometries}
In order for HAPI to do geometry based haptics rendering there must be a way to
specify a geometry. In HAPI these are called shapes.

\section{Available shapes}
The following shapes are general and can be used with any renderer. When
referring to a "primitive" it means that the class inherit from
GeometryPrimitive found in CollisionObjects.h.
\begin{itemize}
\item HapticLineSet - Contains a set of LineSegment primitives.
\item HapticPointSet - Contains a set of Point primitives.
\item HapticPrimitive - Contains one instance of a primitive.
\item HapticPrimitiveSet - Contains a set of primitives.
\item HapticPrimitiveTree - Contains a tree of primitives.
\item HapticTriangleSet - Contains a set of Triangle primitives.
\item HapticTriangleTree - Contains a tree of Triangle primitives
\end{itemize}

The following two shapes are specific for OpenHapticsRenderer and will not
work with any other renderer.
\begin{itemize}
\item HLDepthBufferShape - Use an instance of HAPIGLShape for easy
implementation of a HL\-\_SHAPE\-\_DEPTH\-\_BUFFER in HLAPI by using the render()
function of HAPIGLShape.
\item HLFeedbackShape - Use an instance of a HAPIGLShape for easy
implementation of a HL\-\_SHAPE\-\_FEEDBACK\-\_BUFFER in HLAPI.
\end{itemize}

\section{Using OpenGL to specify shapes}
If you are using OpenGL to do graphics rendering of your objects you
can use the FeedbackBufferCollector class to get hold of the triangles
that are used in your OpenGL graphics rendering and use them for
haptics too. 

\input{examples/feedback_buffer_collector}

The collected triangles could be used with a HapticTriangleSet or structured
in a tree and used with HapticTriangleTree. Then that shape can be sent to
the haptics device to do geometry based haptic rendering.

\section{Creating new shapes}
All geometries sent to be rendered on a haptics device must
inherit from the class HAPIHapticShape. There are a couple of abstract functions
that must be implemented when subclassing HAPIHapticShape.

\begin{itemize}
\item void closestPointOnShape( const Vec3 \&p, Vec3 \&cp, Vec3 \&n, Vec3 \&tc )
- Get the closest point and normal on the object to the given point p.
\item bool lineIntersectShape( const Vec3 \&from, const Vec3 \&to, Collision::IntersectionInfo \&result, Collision::FaceType face )
- Detect collision between a line segment and the object.
\item bool movingSphereIntersectShape( HAPIFloat radius, const Vec3 \&from, const Vec3 \&to )
- Detect collision between a moving sphere and the object.
\item void getConstraintsOfShape( const Vec3 \&point, Constraints \&constraints, Collision::FaceType face, HAPIFloat radius )
- Get constraint planes of the shape. A proxy of a haptics renderer will always stay above any constraints added.
\item void getTangentSpaceMatrixShape( const Vec3 \&point, Matrix4 \&result\_mtx )
- Calculates a matrix transforming a vector from local space of the shape to texture space of the shape.
\item void glRenderShape() - Render a graphical representation of the shape using OpenGL.
\end{itemize}

The first three functions are used for collision detection. The second and
fourth are used by the custom implemented renderers in HAPI. The fifth
function must calculate something useful if DepthMapSurface should work with
the shape. The last function have its uses for debugging purposes and also for
one specific renderer (OpenHapticsRenderer). For more information about the
arguments to the functions see HAPIHapticShape.h.