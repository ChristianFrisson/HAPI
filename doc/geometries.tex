\chapter{Collision geometries}
In order for HAPI to do geometry based haptics rendering there must be a way to
specify a geometry to do this haptic rendering on. In HAPI these collision
geometries are called shapes.

\section{Available shapes}
The following shapes are general and can be used with any renderer although
not all renderers support collision with the shape. RuspiniRenderer is for example the
only implemented renderer for which lines and points can be felt. When
referring to a "primitive" in the list below it means that the class inherit from
GeometryPrimitive found in CollisionObjects.h.

\begin{itemize}
\item HapticLineSet - Contains a set of LineSegment primitives.
\item HapticPointSet - Contains a set of Point primitives.
\item HapticPrimitive - Contains one primitive.
\item HapticPrimitiveSet - Contains a set of primitives.
\item HapticPrimitiveTree - Contains a tree of primitives. The tree used
allows for faster collision detection through grouping of the primitives
into bounding boxes. Check the doxygen documentation (CollisionObject.h)
for more information about the predefined tree classes in HAPI.
\item HapticTriangleSet - Contains a set of Triangle primitives.
\item HapticTriangleTree - Contains a tree of Triangle primitives. The tree used
allows for faster collision detection through grouping of the triangles
into bounding boxes. Check the doxygen documentation (CollisionObject.h)
for more information about the predefined tree classes in HAPI.
\end{itemize}

The following two shapes are specific for OpenHapticsRenderer and will not
work with any other renderer.
\begin{itemize}
\item HLDepthBufferShape - Use an instance of HAPIGLShape for easy
implementation of a HL\-\_SHAPE\-\_DEPTH\-\_BUFFER in HLAPI by using the render()
function of HAPIGLShape.
\item HLFeedbackShape - Use an instance of a HAPIGLShape for easy
implementation of a HL\-\_SHAPE\-\_FEEDBACK\-\_BUFFER in HLAPI.
\end{itemize}

\section{Using OpenGL to specify shapes}
If you are using OpenGL to do graphics rendering of your objects you
can use the FeedbackBufferCollector class to get hold of the triangles
that are used in your OpenGL graphics rendering and use them for
haptics too. 

\input{examples/feedback_buffer_collector}

The collected triangles could be used with a HapticTriangleSet or structured
in a tree and used with HapticTriangleTree. The created shape can be sent to
the haptics device to do geometry based haptic rendering.

\section{Custom made collision geometries}
All geometries sent to be rendered on a haptics device must
inherit from the class HAPIHapticShape. These are the abstract functions
that must be implemented when subclassing HAPIHapticShape.

\begin{itemize}
\item {\ttfamily void closestPointOnShape( const Vec3 \&p, Vec3 \&cp, Vec3 \&n, Vec3 \&tc )}
- Get the closest point and normal on the object to the given point p.
\item {\ttfamily bool lineIntersectShape( const Vec3 \&from, const Vec3 \&to, Collision::IntersectionInfo \&result, Collision::FaceType face )}
- Detect collision between a line segment and the object.
\item {\ttfamily bool movingSphereIntersectShape( HAPIFloat radius, const Vec3 \&from, const Vec3 \&to )}
- Detect collision between a moving sphere and the object.
\item {\ttfamily void getConstraintsOfShape( const Vec3 \&point, Constraints \&constraints, Collision::FaceType face, HAPIFloat radius )}
- Get constraint planes of the shape. A proxy of a haptics renderer will always stay above any constraints added.
\item {\ttfamily void getTangentSpaceMatrixShape( const Vec3 \&point, Matrix4 \&result\_mtx )}
- Calculates a matrix which transforms from local space of the shape to texture space of the shape.
\item {\ttfamily void glRenderShape() - Render a graphical representation of the shape}
using OpenGL. This function only needs to be implemented if HAPI is built with OpenGL support.
\end{itemize}

The first three functions are used for collision detection. The second and
fourth are used by the custom implemented renderers in HAPI. The fifth
function must calculate something useful if DepthMapSurface should work with
the shape. The last function has its uses for debugging purposes and also for
one specific renderer (OpenHapticsRenderer). For more information about the
arguments to the functions see the doxygen documentation.