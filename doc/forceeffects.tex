\chapter{Force effects}
There are two ways to generate forces in HAPI. One is the one
previously described, by specifying shapes and surfaces and letting
a haptics rendering algorithm determine forces when colliding with the
shapes. Another way to generate forces is by using force effects. A
force effect is a global force function specifying what force to
generate at any given moment depending on the current values of the
haptics device. It can, among other things, be used to simulate
viscosity and springs. 

\section{Available force effects}
The force effects available in HAPI are:

\begin{itemize}
\item Force field
\item Spring
\item Viscosity
\item Vibration
\item PlaneConstraint
\item LineConstraint
\item TimeFunctionEffect 
\end{itemize}

\section{Interpolation}
When adding a force effect the user can choose if it should be
interpolated or not. Interpolation of force effects means that newly
added force effects will be ... should we have interpolation? how
should it work??  

\section{Creating new force effects}
A user can easily create their own force effects by subclassing from
HAPIForceEffect and implementing the calculateForces() method. The
method gets the haptics device the force effect is to be rendered on
as argument, allowing it to access all device values. It also get the
time since the last call haptics loop. The output is an en
EffectOutput structure. It contains:

\begin{itemize}
\item force
\item torque
\item position to force jacobian matrix
\item velocity to force jabobian matrix
\end{itemize}

Global coordinates. Transform from local to global coordinates.

E.g. a very simple class generating a constant force would look like.

namespace HAPI {
  /// This is a HAPIForceEffect that generates a constant force.
  class ConstantForce: public HAPIForceEffect {
  public:
    /// Constructor
    ConstantForce( const Vec3 \&\_force ): 
      force( \_force ) {}

    /// The force of the EffectOutput will be the force of the force field. 
    EffectOutput virtual calculateForces( HAPIHapticsDevice *hd,
                                          HAPITime dt ) {
      return EffectOutput( force );
    }
    
  protected:
    Vec3 force;
  };
}



