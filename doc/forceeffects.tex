\chapter{Force effects}
There are two ways to generate forces in HAPI. One is the one
previously described, by specifying shapes and surfaces and letting
a haptics rendering algorithm determine forces when colliding with the
shapes. Another way to generate forces is by using force effects. A
force effect is a global force function which calculates what force to
generate at any given moment. It can, among other things, be used to simulate
viscosity and springs. 

\section{Available force effects}
The force effects available in HAPI are:

\begin{itemize}
\item HapticForceField
\item HapticPositionFunctionEffect
\item HapticShapeConstraint
\item HapticSpring
\item HapticTimeFunctionEffect 
\item HapticViscosity
\end{itemize}

\section{Interpolation}
Force effects does not have an internal property to decide whether they
should be interpolated between frames. If this is needed it is assumed
that this is taken care of by the force calculation itself. However,
when adding or removing a force effect from a haptics device there
is a parameter to make the force effect fade in or out smoothly. See section
\ref{ssHapticsRenderingFunctions} for details.

\section{Creating new force effects}
A user can easily create their own force effects by subclassing from
HAPIForceEffect and implementing the calculateForces() method. The
method takes the struct EffectInput as input. This struct contains:

\begin{itemize}
\item hd - A pointer to the HAPIHapticsDevice on which the force is rendered.
Through this pointer the force effect can access device values such as
the current position, velocity and orientation of the HAPIHapticsDevice.
\item deltaT - The change in time since the last haptic frame.
\end{itemize}
The output of the calculateForces() method is an EffectOutput structure.
It contains:

\begin{itemize}
\item force - The force calculated by the calculateForces() function.
\item torque - The torque calculated by the calculateForces() function.
\end{itemize}

The force and torque outputed should always be given in global coordinates.
Here follows an example of a very simple class that generates a constant force.

\input{examples/HapticForceField}
