
\chapter {HAPI}


\section{Overview}

HAPI consists of the following parts:

\begin{itemize}
\item Device handling
\item Haptics rendering algorithm
\item Collision geometries
\item Surface handling
\item Force effects
\item Thread handling
\end{itemize}


\subsection{Device handling}
The device handling layer provides a device independent
interface for various haptics devices. It covers the following
areas:

\begin{itemize}
\item Device initialisation/cleanup
\item Get position / orientation 
\item Set force / torque
\end{itemize}

HAPI can easily be extended for new devices by implementing abstract
functions for performing those task. See \ref{aa} ??

\subsection{Haptics rendering algorithm}
A haptics rendering algorithm is need in order to calculate forces and
torques from the position and orientation of the device. An interface
is provided to give a user the opportunity to implement their own
algorithm. There are four algorithms already implemented. They are:

\begin{itemize}
\item God object algorithm - point proxy based
\item Ruspini algorithm - sphere proxy based
\item Chai3D - use Chai3D rendering library
\item OpenHaptics - use OpenHaptics rendering library
\end{itemize}

Some of them have restrictions on what features of HAPI can be used,
e.g. some does not support user defined surfaces. Check out the
details of each algorithm to find out what is supported.

\subsection{Collision classes}
HAPI contains a some classes for collision handling used in the
haptics rendering algorithms. A user have to create instances of these
classes and feed them to the haptics device that they are to be
rendered at. It also contains classes for building binary bound trees
such as axis-aligned and oriented bounding box trees from triangles,
that can be used in order to do faster collision detection. 

\subsection{Surface handling}
When touching the surface of a geometry, the haptics rendering
algorithm has to know to what forces to generate depending on the
penetration of the surface. This is handled by the surface classes. A
user can define an arbitrary function for the force and proxy
movement(does not work for all haptics rendering algorithms though as
mentioned earlier).

\subsection{Force effects}
Many times a user wants to generate forces that are not based on
touching geometries, but instead is only depending on the position and
orientation of the haptics device. These are e.g. force fields,
springs, viscosity, etc. In these cases a force effect can be used.

\subsection{Thread handling}
HAPI manages high-priority threads running at 1000 Hz for the haptics
rendering and provides mechanisms to communicate between different
threads. The thread handling functions can also be used to create new
user defined threads.      


