\chapter{Thread handling}

Sometimes it is desirable to do calculations in a thread separate from an
applications main thread. Thread handling in HAPI is needed because haptics
rendering should be done at (at least) 1000 Hz. Therefore each haptics device
needs its own thread for haptics rendering. This chapter covers thread handling
in HAPI.

\section{Thread handling classes}
All classes and everything needed to do proper thread handling in HAPI can be
found in the H3DUtil library. See Threads.h and Threads.cpp (in H3DUtil) for
class declarations and class definitions. Here follows a list and a short
description of what each class is used for:

\begin{itemize}
\item MutexLock - Used to lock/unlock access to data.
\item ConditionLock - A MutexLock with extra features.
\item SimpleThread - The simplest thread possible. Runs one function in a
separate thread.
\item PeriodicThread - Thread has a main loop and it is possible to add and
remove callbacks.
\end{itemize}

\section{Setting up a simple thread}
\label{secSimpleThread}
Setting up a thread using the SimpleThread class is very easy.
The only arguments to SimpleThread are the function that is to be run in a
separate thread and arguments to that function if there are any. If the
function depends on more than one argument these should be collected in a
custom made struct and a pointer to an instance of the struct should be sent
as argument.

A simple example of setting up a thread could be to print "SimpleThread"
over and over again to the console in a separate thread while the main thread
prints "Press ENTER to exit" once to the same console. The main thread will
wait for input from the user and when ENTER is pressed the program will exit.

\input{examples/SimpleThreadPrint_cpp.tex}

\section{Thread safety}
When adding new features which requires data to
be accessed by more than one thread care must be taken so that only one thread
at a time tries to access critical data. In HAPI this can be done in two ways.
One approach is to use callbacks, the other one is to use MutexLocks to lock
access for all other threads except the one that is currently modifying or
reading data. These approaches are not mutually exclusive. It is possible to
mix callbacks and locks.

\subsection{Using locks}
The concept of locks is easy to understand and use. The two most important
functions which exists in both the MutexLock class and ConditionLock class are:

\begin{itemize}
\item lock() - Used to lock access to data.
\item unlock() - Used to release the lock.
\end{itemize}

These two functions are used in pairs to protect code from being accessed by
more than one thread at a time. If a variable is used in two or more threads
there should be a lock/unlock pair around all code that use this variable. If a
function may be called from two or more threads all code in the function should
be enclosed by a lock/unlock pair. It is important to remember that calls to
the lock() function only stop access to data surrounded by lock/unlock pairs
in those places where the same instance of MutexLock (or ConditionLock) is used
to lock. This means that in an application which contains "MutexLock A" and
"MutexLock B" calls to A.lock() will not prevent access to code after calls to
B.lock().

In the example in section \ref{secSimpleThread} one might want to change what
is printed to the console by the SimpleThread without setting up the thread
again. In the following example locks will be used in order to safely change
the content of the string variable used for printing since it is not possible
to set up callbacks for the class SimpleThread. Removing the locks from this
example will most likely crash the program when pressing ENTER. It does not
neccessarily have to happen but there is a high probability of it happening.

\input{examples/SimpleThreadPrintLock_cpp.tex}

\subsection{Using callbacks}
Using callbacks is another way to handle thread safety. All threads that
inherit from the class PeriodicThreadBase must have the following functions
implemented.

\begin{itemize}
\item virtual void synchronousCallback( CallbackFunc func, void *data ) - Adds
  a callback function to be executed in this thread. The calling thread will
  wait until the callback function has returned before continuing.
\item virtual int asynchronousCallback( CallbackFunc func, void *data ) - Adds
  a callback function to be executed in this thread. The calling thread will
  continue executing after adding the callback and will not wait for the
  callback function to execute. Returns a handle to the callback that can be
  used to remove the callback.
\item virtual bool removeAsynchronousCallback( int callback\_handle ) - 
  Attempts to remove a callback. returns true if succeded. returns false if
  the callback does not exist. This function should be handled with care.
  Callbacks are removed if they return CALLBACK\_DONE or a call to this
  function is made.
\end{itemize}

The argument "func" in the above functions contains the callback function to
execute. The callback function has to be of the form
"CallbackCode myCallbackFunc( void *data )" where myCallbackFunc is the name of
the function. It shall return the CallbackCode CALLBACK\_DONE or
CALLBACK\_CONTINUE. As long as the return code is CALLBACK\_CONTINUE the
callback will be called in the next thread loop.

Here follows an example on how to create a thread and set up different
callbacks.

\input{examples/PeriodicThreadCallbacks_cpp.tex}

For thread safety reasons it does not matter if we add the function
printSynchronous as a synchronousCallback or asynchronousCallback. The program
will not crash. The only difference is that the calling thread will wait for
the callback to finish before continuing. However, if the main thread would
change the value of the variable "to\_print" without the use of callback a
crash may occur. At least if the PeriodicThread is set to be called 1000 times
each second.

\section{Threads and haptics devices}
Each HAPIHapticsDevice implemented in HAPI must have a thread in which haptic
rendering is done. The function getThread() returns the thread used by
HAPIHapticsDevice which is a pointer to a PeriodicThreadBase declared in 
Threads.h (in H3DUtil). For this reason the threads used by HAPIHapticsDevice
must be a subclass of PeriodicThreadBase. This means that it is possible to
set up callbacks for the threads of a haptics device in HAPI. By default the
thread used is a HapticThread (see Threads.h).

Most of the time when adding a
new device to HAPI by subclassing HAPIHapticsDevice the default thread is
sufficient. If however the haptics device provides some way of adding callbacks
through calls to other functions a wrapper thread class should be implemented
to override the virtual functions of PeriodicThreadBase for adding and removing
callbacks. For example implementation of this see PhantomHapticsDevice.